%-----------------------------------------------------------------------------
%
%               Template for sigplanconf LaTeX Class
%
% Name:         sigplanconf-template.tex
%
% Purpose:      A template for sigplanconf.cls, which is a LaTeX 2e class
%               file for SIGPLAN conference proceedings.
%
% Guide:        Refer to "Author's Guide to the ACM SIGPLAN Class,"
%               sigplanconf-guide.pdf
%
% Author:       Paul C. Anagnostopoulos
%               Windfall Software
%               978 371-2316
%               paul@windfall.com
%
% Created:      15 February 2005
%
%-----------------------------------------------------------------------------


\documentclass[preprint]{sigplanconf}

% The following \documentclass options may be useful:
%
% 10pt          To set in 10-point type instead of 9-point.
% 11pt          To set in 11-point type instead of 9-point.
% authoryear    To obtain author/year citation style instead of numeric.

\usepackage{amsmath}

\begin{document}

\conferenceinfo{WXYZ '05}{date, City.} 
\copyrightyear{2005} 
\copyrightdata{[to be supplied]} 

\titlebanner{banner above paper title}        % These are ignored unless
\preprintfooter{short description of paper}   % 'preprint' option specified.

\title{An Abstract Virtual Machine for Racket}
\subtitle{(not as object-oriented as it sounds)}

\authorinfo{Name1}
           {Affiliation1}
           {Email1}
\authorinfo{Name2\and Name3}
           {Affiliation2/3}
           {Email2/3}

\maketitle

\begin{abstract}
This is the text of the abstract.
\end{abstract}

\category{CR-number}{subcategory}{third-level}

\terms
term1, term2

\keywords
keyword1, keyword2

\section{Introduction}

The CFA family of analyses [cite k-CFA, gamma-CFA, mu-CFA, etc.] promise to enhance [wc] program efficiency [cite], security [cite], and correctness [cite].

…

Racket programs are organized into modules.
Interaction between different parts of the program is vetted through the module boundary.
An \texttt{import} declaration imports every definition explicitly \texttt{provide}d within the imported module.

While modules are great for managing program complexity, they limit the comprehensiveness of an analysis in the same way … [that inter-procedural analysis enables optimizations otherwise unattainable].
To combat this, modules are first compiled to a high-level intermediate bytecode form and programs are \textit{demodularized} to remove boundaries.
%Finally, a demodularized bytecode program is decompiled to a small set of core forms.
%[move the following discussion]
%These include lambdas, apply-values, define-values, etc.
The compiler performs certain base optimizations such as constant folding [and …] and tracks certain flow information [example], some of which it discovers too late in the compilation process.
%This information is propagated to the decompiler and the resultant core forms are annotated with source information and flow metadata potentially useful to a program optimizer.

\section{Method}

The kernel (not the same as core) language of Racket--the language of the VM--contains over 1000 primitives: arithmetic primitives such as \texttt{+}, \texttt{sqrt}, and \texttt{quotient}; I/O primitives such as \texttt{current-print}, etc. We classify particular subsets of these according to pertinent behaviors.
Included in this is domain checking.
If the analyzer can prove that domain constraints are satisfied in a particular context, the checking can be discarded in all instances of that context.

Are parameters/continuation marks going to require anything special from the analyzer?

\subsubsection{Contract Discharge}
As values conceptually pass through this boundary, dynamic contracts may be attached to assist [wc] blame assignment in specification violations.
Racket's higher-order contracts [cite] allow contracts to be attached to higher-order functions [and dynamically track blame through an arbitrary depth of higher-orderness] but impose a dynamic check at each invocation of the function. 

\section{Racket bytecode}

Instead of a sequence of instructions, Racket bytecode is a tree of bytecode forms.
Literal values are represented directly and there is a fairly close correspondence between high-level core Racket and bytecode.
For example, \texttt{begin} of core Racket becomes \texttt{seq}, \texttt{if} becomes \texttt{branch} and application becomes explicit with \texttt{application}.
\texttt{lambda} is present in the bytecode as \texttt{lam} and augmented with explicit source, parameter, and variable-capture information.


Our goal is to analyze all of Racket, so we must deal with all of its intricacies.

Racket has grown atop a capable VM written in C which provides continuation-handling mechanisms (call/cc), structure things, macro expansion things, etc.



For example, \texttt{letrec} compiles to a \texttt{let-void} form with an \texttt{install-value} subform. [verify]

The extensive VM presents a difficulty in analyzing true and complete Racket. An analyzer must soundly emulate much of the behavior of the VM.

Aside from typical Scheme constructs, Racket has a small set of constructs that provide a base for its vast array of features.
- impersonators and chaperones
- everything in the "Reflection and Security" section

 

\appendix
\section{Appendix Title}

This is the text of the appendix, if you need one.

\acks

Acknowledgments, if needed.

% We recommend abbrvnat bibliography style.

\bibliographystyle{abbrvnat}

% The bibliography should be embedded for final submission.

\begin{thebibliography}{}
\softraggedright

\bibitem[Smith et~al.(2009)Smith, Jones]{smith02}
P. Q. Smith, and X. Y. Jones. ...reference text...

\end{thebibliography}

\end{document}
